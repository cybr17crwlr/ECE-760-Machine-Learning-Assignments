\documentclass[a4paper]{article}
\usepackage{geometry}
\usepackage{graphicx}
\usepackage{natbib}
\usepackage{amsmath}
\usepackage{amssymb}
\usepackage{amsthm}
\usepackage{paralist}
\usepackage{epstopdf}
\usepackage{tabularx}
\usepackage{longtable}
\usepackage{multirow}
\usepackage{multicol}
\usepackage[hidelinks]{hyperref}
\usepackage{fancyvrb}
\usepackage{algorithm}
\usepackage{algorithmic}
\usepackage{float}
\usepackage{paralist}
\usepackage[svgname]{xcolor}
\usepackage{enumerate}
\usepackage{array}
\usepackage{times}
\usepackage{url}
\usepackage{fancyhdr}
\usepackage{comment}
\usepackage{environ}
\usepackage{times}
\usepackage{textcomp}
\usepackage{caption}
\usepackage{multirow}


\urlstyle{rm}

\setlength\parindent{0pt} % Removes all indentation from paragraphs
\theoremstyle{definition}
\newtheorem{definition}{Definition}[]
\newtheorem{conjecture}{Conjecture}[]
\newtheorem{example}{Example}[]
\newtheorem{theorem}{Theorem}[]
\newtheorem{lemma}{Lemma}
\newtheorem{proposition}{Proposition}
\newtheorem{corollary}{Corollary}


\floatname{algorithm}{Procedure}
\renewcommand{\algorithmicrequire}{\textbf{Input:}}
\renewcommand{\algorithmicensure}{\textbf{Output:}}
\newcommand{\abs}[1]{\lvert#1\rvert}
\newcommand{\norm}[1]{\lVert#1\rVert}
\newcommand{\RR}{\mathbb{R}}
\newcommand{\CC}{\mathbb{C}}
\newcommand{\Nat}{\mathbb{N}}
\newcommand{\br}[1]{\{#1\}}
\DeclareMathOperator*{\argmin}{arg\,min}
\DeclareMathOperator*{\argmax}{arg\,max}
\renewcommand{\qedsymbol}{$\blacksquare$}

\definecolor{dkgreen}{rgb}{0,0.6,0}
\definecolor{gray}{rgb}{0.5,0.5,0.5}
\definecolor{mauve}{rgb}{0.58,0,0.82}

\newcommand{\Var}{\mathrm{Var}}
\newcommand{\Cov}{\mathrm{Cov}}

\newcommand{\vc}[1]{\boldsymbol{#1}}
\newcommand{\xv}{\vc{x}}
\newcommand{\Sigmav}{\vc{\Sigma}}
\newcommand{\alphav}{\vc{\alpha}}
\newcommand{\muv}{\vc{\mu}}

\newcommand{\red}[1]{\textcolor{red}{#1}}

\def\x{\mathbf x}
\def\y{\mathbf y}
\def\w{\mathbf w}
\def\v{\mathbf v}
\def\E{\mathbb E}
\def\V{\mathbb V}

% TO SHOW SOLUTIONS, include following (else comment out):
\newenvironment{soln}{
    \leavevmode\color{blue}\ignorespaces
}{}


\hypersetup{
%    colorlinks,
    linkcolor={red!50!black},
    citecolor={blue!50!black},
    urlcolor={blue!80!black}
}

\geometry{
  top=1in,            % <-- you want to adjust this
  inner=1in,
  outer=1in,
  bottom=1in,
  headheight=3em,       % <-- and this
  headsep=2em,          % <-- and this
  footskip=3em,
}


\pagestyle{fancyplain}
\lhead{\fancyplain{}{Homework 3}}
\rhead{\fancyplain{}{CS 760 Machine Learning}}
\cfoot{\thepage}

\title{\textsc{Homework 3}} % Title

%%% NOTE:  Replace 'NAME HERE' etc., and delete any "\red{}" wrappers (so it won't show up as red)

\author{
\red{APOORVA KUMAR} \\
\red{908 461 5997}\\
} 

\date{}

\begin{document}

\maketitle 


\textbf{Instructions:} 
Use this latex file as a template to develop your homework. Submit your homework on time as a single pdf file to Canvas. Late submissions may not be accepted. Please wrap your code and upload to a public GitHub repo, then attach the link below the instructions so that we can access it. You can choose any programming language (i.e. python, R, or MATLAB). Please check Piazza for updates about the homework.\\

\begin{soln}
    \Large
    \href{https://github.com/cybr17crwlr/ECE-760-Machine-Learning-Assignments/tree/master/Homework%203}{Homework 3 Github Link}
\end{soln}

\section{Questions (50 pts)}
\begin{enumerate}
\item (9 pts) Explain whether each scenario is a classification or regression problem. And, provide the number of data points ($n$) and the number of features ($p$).

\begin{enumerate}
	\item (3 pts) We collect a set of data on the top 500 firms in the US. For each firm we record profit, number of employees, industry and the CEO salary. We are interested in predicting CEO salary with given factors.
	
	\begin{soln}
            The following is a \textbf{regression} problem as the CEO salary can be a continous range of numbers.\\
            $n$ = 500 \\
            $p$ = 3 - profit, number of employees, industry
        \end{soln}
	
	\item (3 pts) We are considering launching a new product and wish to know whether it will be a success or a failure. We collect data on 20 similar products that were previously launched. For each product we have recorded whether it was a success or failure, price charged for the product, marketing budget, competition price, and ten other variables.
	
	\begin{soln}
            The following as a binary \textbf{classification} problem with the classes being success or failure of launch.\\
            $n$ = 20 \\
            $p$ = 13
        \end{soln}
	
	\item (3 pts) We are interesting in predicting the \% change in the US dollar in relation to the weekly changes in the world stock markets. Hence we collect weekly data for all of 2012. For each week we record the \% change in the dollar, the \% change in the US market, the \% change in the British market, and the \% change in the German market.
	
	\begin{soln}
            The following is a \textbf{regression} problem as the \% change in market is a continous value.\\
            $n$ = 52 (1 data point per week)\\
            $p$ = 3 - \% change in US market, \% change in British market, \% change in German market
        \end{soln}
	
\end{enumerate}

\item (6 pts) The table below provides a training data set containing six observations, three predictors, and one qualitative response variable.

\begin{center}
	\begin{tabular}{ c  c  c  c}
		\hline
		$X_{1}$ & $X_{2}$ & $X_{3}$ & $Y$ \\ \hline
		0 & 3 & 0 & Red \\
		2 & 0 & 0 & Red \\
		0 & 1 & 3 & Red \\
		0 & 1 & 2 & Green \\
		-1 & 0 & 1 & Green \\
		1 & 1 & 1 & Red  \\
		\hline
	\end{tabular}
\end{center}

Suppose we wish to use this data set to make a prediction for $Y$ when $X_{1} = X_{2} = X_{3} = 0$ using K-nearest neighbors.

\begin{enumerate}
	\item (2 pts) Compute the Euclidean distance between each observation and the test point, $X_{1} = X_{2} = X_{3}=0$.
 
	\begin{soln}
            \begin{center}
                \begin{tabular}{ c  c  c  c| c}
        		\hline
        		$X_{1}$ & $X_{2}$ & $X_{3}$ & $Y$ & \textit{Distance from (0,0,0)} \\ \hline
        		0 & 3 & 0 & Red & \textbf{3} \\
        		2 & 0 & 0 & Red & \textbf{2} \\
        		0 & 1 & 3 & Red & \textbf{3.162} \\
        		0 & 1 & 2 & Green & \textbf{2.236} \\
        		-1 & 0 & 1 & Green & \textbf{1.414} \\
        		1 & 1 & 1 & Red & \textbf{1.732}  \\
        		\hline
        	\end{tabular}
            \end{center}
        \end{soln}
 
	\item (2 pts) What is our prediction with $K=1$? Why?
	
	\begin{soln}
            With K=1 our prediction is \textbf{Green} due to our point, $(0,0,0)$ being closest to $(-1,0,1)$, which belongs to the Green class.
        \end{soln}
	
	\item (2 pts) What is our prediction with $K=3$? Why?
	
	\begin{soln}
            When K=3, the three closest points are $(-1,0,1),\,(1,1,1),\,(2,0,0)$ with classes Green, Red, Red thus we assign our prediction the class \textbf{Red}
        \end{soln}

\end{enumerate}

\item (12 pts) When the number of features $p$ is large, there tends to be a deterioration in the performance of KNN and other local approaches that perform prediction using only observations that are near the test ob- servation for which a prediction must be made. This phenomenon is known as the curse of dimensionality, and it ties into the fact that non-parametric approaches often perform poorly when $p$ is large.

\begin{enumerate}
	\item (2pts) Suppose that we have a set of observations, each with measurements on $p=1$ feature, $X$. We assume that $X$ is uniformly (evenly) distributed on [0, 1]. Associated with each observation is a response value. Suppose that we wish to predict a test observation’s response using only observations that are within 10\% of the range of $X$ closest to that test observation. For instance, in order to predict the response for a test observation with $X=0.6$, we will use observations in the range [0.55, 0.65]. On average, what fraction of the available observations will we use to make the prediction?
	
	\begin{soln}
            We need the expected value of fraction of data-set in 10\% of X. We divide our data-set into three sets with percentage of data-set available for each set:
            \[
            f(X) = \begin{cases}
            100X + 5 & X\leq0.05\\
            10 & 0.05<X\leq0.95\\
            105 - 100X & 0.95<X
            \end{cases}
            \]
            \begin{align*}
                \mathbb{E}(f(X)) &= \int_{0}^{0.05} (100X +5)\,dX + \int_{0.05}^{0.95} 10\,dX +\int_{0.95}^{1} (105 - 100X)\,dX \\
                &= 0.375 + 9 + 0.375\\
                &= 9.75
            \end{align*}
            
            Thus on average we have $\mathbf{9.75\%}$ of data available for making the prediction which is $\mathbf{0.0975}$ of the data.
            
        \end{soln}
	
	
	\item (2pts) Now suppose that we have a set of observations, each with measurements on $p =2$ features, $X1$ and $X2$. We assume that predict a test observation’s response using only observations that $(X1,X2)$ are uniformly distributed on [0, 1] × [0, 1]. We wish to are within 10\% of the range of $X1$ and within 10\% of the range of $X2$ closest to that test observation. For instance, in order to predict the response for a test observation with $X1 =0.6$ and $X2 =0.35$, we will use observations in the range [0.55, 0.65] for $X1$ and in the range [0.3, 0.4] for $X2$. On average, what fraction of the available observations will we use to make the prediction?
	
	\begin{soln}
            As the set of observations are uniformly distributed along the whole observation space we can say that $X_1$ is independent of $X_2$. Thus the expectation across the two dimensions is the product of individual expected percentage of observation available.\\
            Therefore average fraction of observations available = $0.0975\times 0.0975 = \mathbf{0.00950625}$
        \end{soln}
	
	\item (2pts) Now suppose that we have a set of observations on $p = 100$ features. Again the observations are uniformly distributed on each feature, and again each feature ranges in value from 0 to 1. We wish to predict a test observation’s response using observations within the 10\% of each feature’s range that is closest to that test observation. What fraction of the available observations will we use to make the prediction?
	
	\begin{soln}
            Using the same theory as in the last question we get:\\
            Avg. Fraction of data available  = $0.0975^{100} = \mathbf{7.952\times10^{-102}}$
        \end{soln}
	
	\item (3pts) Using your answers to parts (a)–(c), argue that a drawback of KNN when p is large is that there are very few training observations “near” any given test observation.
	
	\begin{soln}
            We can see that for a $p$ dimensional observation set, the average number of training samples "near" the test data is $= \mathbf{0.0975^p}$\\
            Consequently, as $p$ increases, the total observations picked for predicting the test data \textbf{decrease exponentially} and \textbf{tend to zero}. This, in turn, doesn't consider the different relations between the data, which may be captured in regions "not-near" the test data and hence is not appropriately modeled. 
        \end{soln}
	
	\item (3pts) Now suppose that we wish to make a prediction for a test observation by creating a $p$-dimensional hypercube centered around the test observation that contains, on average, 10\% of the training observations. For $p =$1, 2, and 100, what is the length of each side of the hypercube? Comment on your answer.
	
	\begin{soln}
            For a $p$ dimensional data the hypercube of length $a$ covers the region $a^p$.\\
            Hence,
            \[
                a^p = 0.1 \Rightarrow a = 0.1^{1/p}
            \]
            For $p=1$,
            \[
                a = 0.1
            \]
            For $p=2$,
            \[
                a = \sqrt{0.1} = 0.316
            \]
            For $p=100$,
            \[
                a = 0.1^{1/100} = 0.977
            \]
            We can see that as the dimension of data increases, the observation set along any one dimension, that we pick for classification, increases drastically.
        \end{soln}
	
\end{enumerate}

\item (6 pts) Supoose you trained a classifier for a spam detection system. The prediction result on the test set is summarized in the following table.
\begin{center}
	\begin{tabular}{l l | l l}
		&          & \multicolumn{2}{l}{Predicted class} \\
		&          & Spam           & not Spam           \\
		\hline
		\multirow{2}{*}{Actual class} & Spam     & 8              & 2                  \\
		& not Spam & 16             & 974               
	\end{tabular}
\end{center}

Calculate
\begin{enumerate}
	\item (2 pts) Accuracy
	\begin{soln}$\frac{TP+TN}{TP+TN+FP+FN} = \frac{982}{1000} = 0.982$\end{soln}
	\item (2 pts) Precision
	\begin{soln}$\frac{TP}{TP+FP} = \frac{8}{24} = 0.333$\end{soln}
	\item (2 pts) Recall
	\begin{soln}$\frac{TP}{TP+FN} = \frac{8}{10} = 0.8$\end{soln}
\end{enumerate}


\item (9pts) Again, suppose you trained a classifier for a spam filter. The prediction result on the test set is summarized in the following table. Here, "+" represents spam, and "-" means not spam.

\begin{center}
\begin{tabular}{ c  c }
\hline
Confidence positive & Correct class \\ \hline
0.95 & + \\
0.85 & + \\
0.8 & - \\
0.7 & + \\
0.55 & + \\
0.45 & - \\
0.4 & + \\
0.3 & + \\
0.2 & - \\
0.1 & - \\
\hline
\end{tabular}
\end{center}

\begin{enumerate}
	\item (6pts) Draw a ROC curve based on the above table.
	
	\begin{soln}
            \begin{figure}
                \centering
                \includegraphics[scale=0.4]{Images/ROC.png}
                \caption{ROC Curve}
                \label{fig:roc_curve}
            \end{figure}
        \end{soln}
	
	\item (3pts) (Real-world open question) Suppose you want to choose a threshold parameter so that mails with confidence positives above the threshold can be classified as spam. Which value will you choose? Justify your answer based on the ROC curve.
	
	\begin{soln}
            Technically the point closest to (0,1) is chosen as the best threshold for the estimator hence here we will choose the point with confidence of 0.45.\\
            But we should also consider the fact that important mails shouldn't be classified as spam whereas the inverse of that isn't much of an issue. Thus following that logic the point where we encounter our first false positive should be a threshold giving us a threshold of 0.8.\\
            I feel a threshold of \textbf{0.8} is better for a safer model.
        \end{soln}
\end{enumerate}

\item (8 pts) In this problem, we will walk through a single step of the gradient descent algorithm for logistic regression. As a reminder,
$$f(x;\theta) = \sigma(\theta^\top x)$$
$$\text{Cross entropy loss } L(\hat{y}, y) = -[y \log  \hat{y} + (1-y)\log(1-\hat{y})]$$
$$\text{The single update step } \theta^{t+1} = \theta^{t} - \eta \nabla_{\theta} L(f(x;\theta), y) $$



\begin{enumerate}
	\item (4 pts) Compute the first gradient $\nabla_{\theta} L(f(x;\theta), y)$.
	
	\begin{soln}
            \begin{align*}
                L(f(x;\theta), y) &= -[y \log f(x;\theta) + (1-y)\log(1-f(x;\theta))]\\
                &= -[y \log(\sigma(\theta^\top x)) + (1-y)\log(1-\sigma(\theta^\top x))]\\
            \text{Therefore,}\\
                \nabla_{\theta} L(f(x;\theta), y) &= -\nabla_{\theta}[y \log(\sigma(\theta^\top x)) + (1-y)\log(1-\sigma(\theta^\top x))]\\
                &= -[y \frac{\sigma'(\theta^\top x)}{\sigma(\theta^\top x)} x + (1-y) \frac{-\sigma'(\theta^\top x)}{1-\sigma(\theta^\top x)} x]\\
            \text{Now, we know}\\
                \quad\sigma'(\theta^\top x) &= \sigma(\theta^\top x)(1-\sigma(\theta^\top x))\\
            \text{Therefore,}\\
                \nabla_{\theta} L(f(x;\theta), y) &= -[y \frac{\sigma(\theta^\top x)(1-\sigma(\theta^\top x))}{\sigma(\theta^\top x)} x + (1-y) \frac{-\sigma(\theta^\top x)(1-\sigma(\theta^\top x))}{1-\sigma(\theta^\top x)} x]\\
                &= - y(1-\sigma(\theta^\top x))x + (1-y)\sigma(\theta^\top x)x\\
                &= -yx + y\sigma(\theta^\top x)x + \sigma(\theta^\top x)x - y\sigma(\theta^\top x)x\\
                &= \mathbf{(\sigma(\theta^\top x) - y)x}\\
                &= \mathbf{(f(x;\theta) - y)x}
            \end{align*}
        \end{soln}
	
	\item (4 pts)
 Now assume a two dimensional input. After including a bias parameter for the first dimension, we will have $\theta\in\mathbb{R}^3$.
$$ \text{Initial parameters : }  \theta^{0}=[0, 0, 0]$$
$$ \text{Learning rate }\eta=0.1$$
$$ \text{data example : } x=[1, 3, 2], y=1$$
Compute the updated parameter vector $\theta^{1}$ from the single update step.
	
	\begin{soln}
            \begin{gather*}
                \theta^{0\top} x = 0\\
                f(x;\theta^0) = \hat{y} = 0.5\\
                \nabla_{\theta} L(f(x;\theta), y) = (0-0.5)\times[1, 3, 2] = [-0.5, -1.5, -1]\\
                \theta^{1} = \theta^{0} - \eta \nabla_{\theta} L(f(x;\theta), y) = [0,0,0] - 0.1\times[-0.5, -1.5, -1] = [0.05, 0.15, 0.1]\\
                \mathbf{\theta^{1} = [0.05, 0.15, 0.1]}
            \end{gather*}
        \end{soln}
\end{enumerate}
\end{enumerate}

\section{Programming (50 pts)}
\begin{enumerate}
	\item (10 pts) Use the whole D2z.txt as training set.  Use Euclidean distance (i.e. $A=I$).
	Visualize the predictions of 1NN on a 2D grid $[-2:0.1:2]^2$.
	That is, you should produce test points whose first feature goes over $-2, -1.9, -1.8, \ldots, 1.9, 2$, so does the second feature independent of the first feature.
	You should overlay the training set in the plot, just make sure we can tell which points are training, which are grid.
    \begin{soln}
        \begin{figure}[h]
		\centering
		\includegraphics[scale=0.4]{Images/1NN.png}
            \caption{1NN Decision Boundary}
            \label{fig:1nndb}
	\end{figure}
    \end{soln}
	
	The expected figure looks like this.
	\begin{figure}[h]
		\centering
		\includegraphics[scale=0.12]{Images/Original/implementation1_expected_result.png}
	\end{figure}
	
	\paragraph{Spam filter} Now, we will use 'emails.csv' as our dataset. The description is as follows.
	\begin{figure}[H]
		\centering
		\includegraphics[width=\linewidth]{Images/Original/email_head.png}
	\end{figure}
	
	\begin{itemize}
		\item Task: spam detection
		\item The number of rows: 5000
		\item The number of features: 3000 (Word frequency in each email)
		\item The label (y) column name: `Predictor'
		\item For a single training/test set split, use Email 1-4000 as the training set, Email 4001-5000 as the test set.
		\item For 5-fold cross validation, split dataset in the following way.
		\begin{itemize}
			\item Fold 1, test set: Email 1-1000, training set: the rest (Email 1001-5000)
			\item Fold 2, test set: Email 1000-2000, training set: the rest
			\item Fold 3, test set: Email 2000-3000, training set: the rest
			\item Fold 4, test set: Email 3000-4000, training set: the rest
			\item Fold 5, test set: Email 4000-5000, training set: the rest			
		\end{itemize}
	\end{itemize}
	
	\item (8 pts) Implement 1NN, Run 5-fold cross validation. Report accuracy, precision, and recall in each fold.
	
	\begin{soln} 
            \begin{center}
                \begin{tabular}{c|c c c}
                    \hline
                    Fold & Accuracy & Precision & Recall \\
                    \hline
                    1	& 0.825 & 0.654 & 0.818\\
                    2	& 0.853 & 0.686 & 0.866\\
                    3	& 0.862 & 0.721 & 0.838\\
                    4	& 0.851 & 0.716 & 0.816\\
                    5	& 0.775 & 0.606 & 0.758\\
                \end{tabular}
            \end{center}
            
        \end{soln}
	
	\item (12 pts) Implement logistic regression (from scratch). Use gradient descent (refer to question 6 from part 1) to find the optimal parameters. You may need to tune your learning rate to find a good optimum. Run 5-fold cross validation. Report accuracy, precision, and recall in each fold.
	
	\begin{soln}
            \begin{center}
                \begin{tabular}{c|c c c}
                    \hline
                    Fold & Accuracy & Precision & Recall \\
                    \hline
                       1	  & 0.891,	  & 0.819,	  & 0.793\\
                       2	  & 0.676,	  & 0.46,	  & 0.968\\
                       3	  & 0.865,	  & 0.89,	  & 0.599\\
                       4	  & 0.832,	  & 0.652,	  & 0.918\\
                       5	  & 0.725,	  & 0.529,	  & 0.912\\
                \end{tabular}
            \end{center}
        \end{soln}
	
	\item (10 pts) Run 5-fold cross validation with kNN varying k (k=1, 3, 5, 7, 10). Plot the average accuracy versus k, and list the average accuracy of each case. \\
        \begin{soln}
            \begin{figure}[H]
    		\centering
    		\includegraphics[scale=0.5]{Images/KNN.png}
                \caption{KNN 5-Fold Cross val for K=1,3,5,7,10}
                \label{fig:knndb}
    	\end{figure}

            \begin{center}
                \begin{tabular}{c|c}
                    \hline
                    k & Avg. Accuracy \\
                    \hline
                       1 & 0.8332\\
                       3 & 0.8422\\
                       5 & 0.8408\\
                       7 & 0.8462\\
                       10 & 0.8556\\
                \end{tabular}
            \end{center}
        \end{soln}
 
	Expected figure looks like this.
	\begin{figure}[H]
		\centering
		\includegraphics[width=8cm]{Images/Original/knn.png}
	\end{figure}
	
	\item (10 pts) Use a single training/test setting. Train kNN (k=5) and logistic regression on the training set, and draw ROC curves based on the test set. \\
        \begin{soln}
            \begin{figure}[H]
    		\centering
    		\includegraphics[scale=0.5]{Images/ROC_Final.png}
                \caption{ROC Curves for 5-NN and Logistic Regression}
                \label{fig:rocknnlr}
    	\end{figure}
        \end{soln}
	Expected figure looks like this.
	\begin{figure}[h]
		\centering
		\includegraphics[width=8cm]{Images/Original/roc.png}
	\end{figure}
	Note that the logistic regression results may differ.
	
	
	
\end{enumerate}
\bibliographystyle{apalike}
\end{document}
